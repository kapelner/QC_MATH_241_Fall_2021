\documentclass[12pt]{article}

\include{preamble}

\newtoggle{spacingmode}
\toggletrue{spacingmode}  %STUDENTS: DELETE or COMMENT this line

\newtoggle{professormode}
\toggletrue{professormode} %STUDENTS: DELETE or COMMENT this line

\newcommand{\spc}[1]{\iftoggle{spacingmode}{\\ \vspace{#1cm}}}


\title{MATH 241 Fall \the\year~Homework \#2 INCOMPLETE}

\author{Professor Adam Kapelner} % STUDENTS: DELETE my name and put your name and section here e.g. \author{John Doe, Section A}. MAKE SURE YOU PUT YOUR SECTION HERE!!!!!!!!

\iftoggle{professormode}{
\date{Due 11:59PM by email, ??? \\ \vspace{0.5cm} \small (this document last updated \today ~at \currenttime)}
}


\renewcommand{\abstractname}{Instructions and Philosophy}

\begin{document}
\maketitle

\iftoggle{professormode}{
\begin{abstract}
The path to success in this class is to do many problems. Unlike other courses, exclusively doing reading(s) will not help. Coming to lecture is akin to watching workout videos; thinking about and solving problems on your own is the actual ``working out''.  Feel free to \qu{work out} with others; \textbf{I want you to work on this in groups.}

Reading is still \textit{required}. For this homework set, read the section about sample spaces in Chapter 2 and relevant parts of Chapter 1 in Ross. Chapter references are from the 7th edition.

The problems below are color coded: \ingreen{green} problems are considered \textit{easy} and marked \qu{[easy]}; \inorange{yellow} problems are considered \textit{intermediate} and marked \qu{[harder]}, \inred{red} problems are considered \textit{difficult} and marked \qu{[difficult]} and \inpurple{purple} problems are extra credit. The \textit{easy} problems are intended to be ``giveaways'' if you went to class. Do as much as you can of the others; I expect you to at least attempt the \textit{difficult} problems.

This homework is worth 100 points but the point distribution will not be determined until after the due date. See syllabus for the policy on late homework.

Up to [see syllabus] points are given as a bonus if the homework is typed using \LaTeX. Links to instaling \LaTeX~and program for compiling \LaTeX~is found on the syllabus. You are encouraged to use \url{overleaf.com}. If you are handing in homework this way, read the comments in the code; there are two lines to comment out and you should replace my name with yours and write your section. If you are asked to make drawings, you can take a picture of your handwritten drawing and insert them as figures or leave space using the \qu{$\backslash$vspace} command and draw them in after printing or attach them stapled.

The document is available with spaces for you to write your answers. If not using \LaTeX, print this document and write in your answers. I do not accept homeworks not on this printout. Keep this first page printed for your records. Write your name below.

\end{abstract}

\thispagestyle{empty}
\vspace{1cm}
NAME: \line(1,0){240} %~~SECTION (A or B): \line(1,0){35}
\pagebreak
}

~\\ \\


\problem Examine the following words and tell me how many \textit{permutations} there are of the letters. We do not care about keeping track of the individual common letters. For example, in the word $dad$, there are two $d's$ and we want to treat the permutation $d_1 d_2 a$ the \textit{same} as $d_2 d_1 a$.

\begin{enumerate}
\easysubproblem town \spc{1}

\easysubproblem tsktsk (yes, this is a real word!)\spc{1}

\intermediatesubproblem mississippi \spc{2.3}

\hardsubproblem supercalifragilisticexpialidocious \spc{3.0} 

\end{enumerate}

\problem Below is a standard chessboard. Rows one and eight have the following pieces: two rooks, two knights, two bishops, a king and a queen. Rows two and seven have 8 pawns. Rows one and two have all black pieces and rows seven and eight have all white pieces.

\iftoggle{professormode}{
\begin{figure}[htp]
\centering
\includegraphics[width=2.5in]{chess.jpg}
\end{figure}
\FloatBarrier
}

\begin{enumerate}
\easysubproblem How many ways are there to place the black queen on a white square? \spc{1}

\intermediatesubproblem How many ways are there to set up the pieces in the back ranks of both white and black \ie arrange the two rooks, two knights, two bishops, king and queen on the first row of 8 squares. Note that this game is called ``Fischer Random Chess'' after the famous grandmaster Bobby Fischer who proposed the idea to make standard chess more exciting.  \spc{2}

\hardsubproblem The game progresses and white takes two black pawns and black takes two white pawns. How many ways are there to arrange the pieces on the board? We don't care about pieces of a type being unique (\ie all white pawns are the same, all black rooks are the same, etc).  \spc{4}

\hardsubproblem Are all arrangements \qu{equally likely} during an actual chess game? Explain why or why not. \spc{3}

\end{enumerate}

\end{document}
